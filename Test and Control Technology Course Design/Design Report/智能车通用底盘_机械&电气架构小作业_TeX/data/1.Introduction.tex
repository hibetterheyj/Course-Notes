\newpage
%——————————————————————设置开始计数———————————————————
\setcounter{page}{1}
\cfoot{\thepage}

\section{绪论}
\subsection{研究背景与意义}

近两年,自动驾驶技术受到越来越多人的关注。自动驾驶(英语:Autopilot)是一种经由机械、电子仪器、液压系统、陀螺仪等,做出无人操控的自动化驾驶。常用在飞行器、船舰及部分的铁路列车。公路交通工具的自动驾驶仍在研究开发中,尚未大规模商用。

而AGV 是 Automated Guided Association 的简称,是一种以充电电池为动力,自动导引的无人驾驶自动化车辆,它能在计算机的监控下,按路径规划和作业要求,精确行走并停靠到指定的地点,完成一系列的作业任务如取货、送货、充电等。

自动驾驶在商用车等领域的与运用与自动导引小车存在很大的相似性,但是在应用场景、数据量大小存在一定程度上的差异,不过在传感器与信息处理系统架构和总线布置相互之间存在很大的相互借鉴空间。本文将从AGV、自动驾驶架构等的基础上出发,去进行通用无人车底盘的机械设计与电控布线,并进行可行性与效益分析,最终希望设计一款适用于大小型自动驾驶车辆的、模块化底盘。

\subsection{AGV国内外研究发展综述}

\subsubsection{无人车辆技术进展}

最近在自动驾驶车辆发展方面的技术进步被认为将带来移动交通领域的下一次革命,它的引入有望彻底改变移动和运输物流的处理方式。

在过去的几年中,一些自动驾驶车辆已经在学术和工业研究领域得到发展。 第一个已知的自主车辆竞赛是DARPA Grand Challenge[1],它激励了近十年来自动驾驶车辆的发展。

说到AGV自动导航小车,产生于 20 世纪 50 年代,是一个集环境感知、规划决策、智能控制等功能为一体的复杂系统,是一种无人驾驶的自动化运输车辆,又可称为轮式移动机器人(Wheeled Mobile Robot,WMR)。一套成熟的 AGV 系统能够根据环境感知传感器获取周围的环境信息建立自主导航的环境地图进行定位,并且能够利用导航地图进行相应的路径规划或进行指定预设路径行驶,完成货物的自动化运输等任务,是实现现代化生产制造、仓库自动化运输、餐厅现代化等重要设备,同时 AGV 能够在环境恶劣、危险场所进行工作。

随着科技的迅速发展,生产、生活过程中的智能化水平不断提高,传统的人力机械劳动将逐渐被 AGV 等智能化设备所替代。而在近十年由于传感器价格的降低以及普及加之处理器与相关技术愈加成熟,也涌现出很多科研上的进展:
国内消费级无人机领导者大疆也先后两年在ICRA会议期间,举办AGV机器人比赛,主要以AGV底盘并增加额外功能(如抓取码垛、如自动射击等)来进行相关竞赛,一定程度上促进了来自交叉学科对于该领域的关注与认识。

%[1]: Iagnemma K, Buehler M. Special issue on the DARPA Grand Challenge[J]. 2006, 23(8):461-462. 
%[2]: DJI Robomaster 2018 ICRA https://www.robomaster.com/en-US/robo/2018

\subsubsection{感知与规划技术进展}

自主导航技术作为 AGV 核心技术之一,是 AGV 技术发展水平的最重要标志。在AGV 自主导航技术的研究中,AGV 车辆对于导航环境中障碍物的检测是能否成功避开障碍物到达指定目标点的重要因素,其关键在于如何基于传感器技术及环境感知技术将获取到的导航环境中障碍物的距离和方位进行处理,并决策出车辆前进的最佳方向及行驶速度。目前,超声波传感器、视觉传感器、红外传感器、激光传感器等广泛应用于AGV 车辆的环境感知技术。

而路径规划是指车辆在导航区域内,给出车辆行驶的起始点和目标点,利用车辆上搭载的环境感知传感器获取环境信息,规划决策出一条从起始点到目标点满足一定性能指标的最佳路径。随着路径规划技术的深入研究,国内外涌现出大量的路径规划算法。根据获得的环境地图信息完整程度,可以将路径规划算法分为全局路径规划算法和局部避障算法。 

\subsubsection{SLAM 技术的研究进展}

SLAM的英文全称是Simultaneous Localization and Mapping,中文称作“同时定位与地图创建”,试图解决的是这样的问题:一个机器人在未知的环境中运动,如何通过对环境的观测确定自身的运动轨迹,同时构建出环境的地图。

SLAM技术正是为了实现这个目标涉及到的诸多技术的总和。因此,SLAM技术涵盖的范围非常广,按照不同的传感器、应用场景、核心算法,SLAM有很多种分类方法。按照传感器的不同,可以分为基于激光雷达的 2D/3D SLAM、基于深度相机的RGBD SLAM、基于视觉传感器的visual SLAM(以下简称vSLAM)、基于视觉传感器和惯性单元的visual inertial odometry(以下简称VIO)。

基于激光雷达的2D SLAM相对成熟,早在2005年,Sebastian Thrun等人的经典著作《概率机器人学》将2D SLAM研究和总结得非常透彻,基本确定了激光雷达SLAM的框架。目前常用的Grid Mapping方法也已经有10余年的历史。2016年,Google开源了激光雷达SLAM程序Cartographer,可以融合IMU信息,统一处理2D与3D SLAM 。目前2D SLAM已经成功地应用于扫地机器人中。

在刚刚结束的ICRA2018中来自苏黎世联邦理工的AMZ Driveless团队为我们展示了在2017年参加Formula Student Driveless斩获冠军所借助的以LiDAR与双目摄像头等多传感器SLAM信息融合实现特定场景(由桩桶组成的赛道)重构后高速行驶的技术[2]。

\subsection{AGV国内应用现状}

AGV 车辆在国内的研究和应用的起步较晚,开始主要由国内一些知名大学和研究所对 AGV 车辆展开研究。吉林大学智能车辆课题组根据一汽汽车装配生产线的要求,成功研制了用于汽车装配生产的 AGV 车辆。清华大学研制的 THMR-V 车辆能够自动跟踪车道线并能自动躲避障碍物。同时,国内涌现了许多成功的 AGV 企业,其中最具代表性的企业为沈阳新松机器人自动化股份有限公司和云南昆船智能装备有限公司。沈阳新松机器人公司在 AGV 车辆的机械设计、导航控制以及多 AGV 管理方面取得了较高的成就。

而在 \textbf{特定场景无人驾驶车辆},近几年国内大型电商物流等公司开始对包括无人驾驶物流车辆、快递车辆、清扫车辆、园区运货车辆、景区观光和摆渡车辆等进行积极的探索。因此专用化的AGV被公认为是无人驾驶技术目前最具有商业落地前景的应用场景。近年内,京东、菜鸟网络、苏宁物流等物流、电商巨头,以及智行者、主线科技、驭势科技等初创企业都在特定场景无人驾驶车辆上迅速发力。而下表1.1就是近三年特定场景无人车辆典型应用发展概况:

\begin{table}[htbp]
	\centering%表格居中
	\caption[centering]{特定场景无人车辆典型应用发展概况}%表格标题
	\label{共享单车日均使用总量与对应时间段的共享单车数量}%表格标签
	\begin{tabular}{C{2cm}p{10cm}}	
		\toprule
		\tabincell{c}{\textbf{企业名称}} &\tabincell{c}{\textbf{\hspace{12em} 发展情况}} \\ 
		\midrule
		\textbf{京东} & 2016年9月,宣布无人驾驶物流车开始路测。2017年6月,在北京、西安、杭州等6所高校内同时试运营。2017年下半年,发布无人驾驶物流车3.0版。 \\
		\\
		\textbf{菜鸟网络} & 2016年9月,宣布无人驾驶物流车开始路测。2017年6月,在北京、西安、杭州等6所高校内同时试运营。2017年下半年,发布无人驾驶物流车3.0版。 \\
		\\
		\textbf{苏宁物流} & 2018年3月,发布无人驾驶物流车“卧龙一号”,并在南京某社区试运营。 \\
		\\
		\textbf{驭势科技} & 2016年2月成立, 11月宣布完成B轮融资。2017年3月发布无人驾驶机场摆渡车,并在广州白云机场试运营。 \\
		\\
		\textbf{仙途智能} & 2017年7月成立,2018年3月宣布完成A轮融资,发布无人驾驶清扫车队,并在上海某科技园区试运营。\\
		\bottomrule
	\end{tabular}
\end{table}

目前,采用低成本的电动代步车底盘改装成为了现行主要解决方案,但其行驶速度、机动能力、载重能力和平顺性等较差。同时,电动代步车底盘本身是面向有人驾驶的,其存在着诸多问题,如转向盘等机械结构冗余、线控化程度低、续航能力差、电池系统不满足车规级需求、控制系统粗糙、执行元件精度低等,难以满足无人驾驶车辆实际的性能需求。更重要的是,当前用于改装的底盘五花八门、改装方案良莠不齐,这严重制约了未来特定场景无人驾驶车辆的网联化监测与运营。

\subsection{设计目标与意义}

\subsubsection{主要背景与方向}

希望以推动特定场景无人驾驶车辆大范围落地为近期目标,致力于无人驾驶车辆车规级“通用线控底盘”研发,为特定场景无人驾驶车辆用户提供通用的智能底盘平台;远景目标瞄向构建以通用底盘平台为基础的全工况、全维度、全周期的特定场景无人驾驶车辆监控与运营的大数据平台。

“自动车通用底盘”:以“通用化”为核心特点,底盘被封装在一个扁平的车身中,依据不同造型与功能需求,可搭载不同上装功能模块,如物流、快递、清扫、运输、甚至军警用特种装备等模块,成为适用于特定场景下各种功能的无人驾驶车辆。

\subsubsection{具体技术要求}

具体尺寸不超过 $ 1000mm \times 1000mm $的尺寸,以麦克纳姆轮或万向轮为基础,实现简易有效的避震设计,传感器选用上采取模块化方法,并尽可能选取标准件,保障整体系统稳定性的实现。

\subsection{本报告章节安排}

本报告从AGV自动化小车缘由谈起,结合AGV在研究与应用现状与进展,收集整理了国内外在自动化导航小车的应用与不足,并比较分析后引出主要设计的目标与方向;接着就所提出的要求分别进行机械、电控从总体到模块分析的选型、设计,最后就整个机器人实现层面进行重难点分析与经济社会分析后,最终进行课程总结,主要内容包括以下几个部分:

第一章是主要分析与介绍AGV自动化小车的发展与应用前景,提出设计要求,并说明主要章节安排;

第二章是就所提出要求进行机械部分的总体结构设计,首先是功能模块分析概念设计,然后进行具体模块设计,最后进行模型与场景呈现;

第三章是结合第二章就所提出的预期目标以及第二章节所设计的机器人的基础上进行电气总线上的设计,主要从架构设计、功能分析、传感器选型以及软件的预期实现进行陈述;

第四章是在本报告的设计目标基础上,从在前两章进行的设计出发,在技术重难点以及经济社会效益两方面进行可行性分析;

第五章是本次设计报告的感想与关于《测试与控制技术课程设计》的总结与反馈;

最后部分为本文参考文献。