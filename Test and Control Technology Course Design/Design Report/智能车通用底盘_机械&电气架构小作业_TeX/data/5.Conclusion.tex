\newpage
\section{总结感想}

自从16年秋天加入同济Super Power RoboMaster机器人战队将近两年了,我的身份也从一个小弟慢慢成长为一个负责人与规则解读者,很欣喜的看到队伍从去年的崭露头角最后折戟踢馆赛到如今能够在中部赛区进入四强,着实十分激动与欣慰,很感谢一路走来的队友,让我们共同感受这比赛的精彩,也让我对于机械电控有较多的认识并且深深意识到自己的不足以及上升空间。

本学期很有幸能够选择这节由符长虹老师教授的测试与控制技术课程设计,在课上我们感受的前所未有的技术带给人的那种心潮澎湃,让我深深意识到自己在电控、算法、编程以及诸多方向的不足,不过这也促使我下定决心想要去做一些前沿的科研,走过去没走而应该走的路,去迎接更大的挑战。因此借测控课设之际,在符老师的鼓励和引导下,结合在RM所作,以及自己即将到来的暑期实习,头一次将机器人底盘从机械设计到电控框架的设计的整体流程从设计、建模再到选型等流程走下来一遍,并整理形成一份报告。尽管从设计到最终实现还有一定的距离,况且报告有很多不完善之处,但相信经过接下去一段时间的努力,多积累知识、多尝试、多做科研,为自己的梦想去奋斗,努力将自己塑造成一个真正具有工程素养、创新思维与人文精神兼具的机器人学家吧!再次感想来自老师、同学与朋友的解惑、讨论与互相促进!

\begin{flushright}
您的学生

何宇杰

18.06.21
\end{flushright}