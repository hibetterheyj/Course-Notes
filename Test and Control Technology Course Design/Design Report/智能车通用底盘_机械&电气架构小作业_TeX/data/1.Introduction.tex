\newpage
\section{绪论}
\subsection{研究背景与意义}

AGV 是 Automated Guided Association 的简称,是一种以充电电池为动力,自动导引的无人驾驶自动化车辆,它能在计算机的监控下,按路径规划和作业要求,精确行走并停靠到指定的地点,完成一系列的作业任务如取货、送货、充电等。
AGV 是移动机器人的一个重要分支,以应用为目的的 AGV,又称为自主式无人搬运车。其研究重点是实际工业生产中可能面临的问题,如: AGV 控制器组合结构的设计、基于电子地图的AGV 运行路线避碰调度、任务调度的应用等。

\subsection{AGV研究现状}



\subsection{AGV应用现状}

\begin{table}[htbp]
	\centering%表格居中
	\caption[centering]{特定场景无人车辆典型应用发展概况}%表格标题
	\label{共享单车日均使用总量与对应时间段的共享单车数量}%表格标签
	\begin{tabular}{C{2cm}p{10cm}}	
		\toprule
		\tabincell{c}{\textbf{企业名称}} &\tabincell{c}{\textbf{\hspace{12em} 发展情况}} \\ 
		\midrule
		\textbf{京东} & 2016年9月,宣布无人驾驶物流车开始路测。2017年6月,在北京、西安、杭州等6所高校内同时试运营。2017年下半年,发布无人驾驶物流车3.0版。 \\
		\\
		\textbf{菜鸟网络} & 2016年9月,宣布无人驾驶物流车开始路测。2017年6月,在北京、西安、杭州等6所高校内同时试运营。2017年下半年,发布无人驾驶物流车3.0版。 \\
		\\
		\textbf{苏宁物流} & 2018年3月,发布无人驾驶物流车“卧龙一号”,并在南京某社区试运营。 \\
		\\
		\textbf{智行者} & 2015年5月成立,2017年发布无人驾驶物流车和无人驾驶清扫车,并在清华大学、奥森公园等多地试运营。2018年4月宣布完成B1、B2轮融资。 \\
		\\
		\textbf{主线科技} & 2017年3月成立,9月宣布完成天使轮融资。2018年4月发布全球首台无人驾驶港口用电动卡车,并在天津港试运营。 \\
		\\
		\textbf{驭势科技} & 2016年2月成立, 11月宣布完成B轮融资。2017年3月发布无人驾驶机场摆渡车,并在广州白云机场试运营。 \\
		\\
		\textbf{NURO.ai} & 2016年成立,融资情况不详。2018年1月,在美国硅谷发布无人驾驶物流车。 \\
		\\
		\textbf{仙途智能} & 2017年7月成立,2018年3月宣布完成A轮融资,发布无人驾驶清扫车队,并在上海某科技园区试运营。\\
		\bottomrule
	\end{tabular}
\end{table}

目前,采用低成本的电动代步车底盘改装成为了现行主要解决方案,但其行驶速度、机动能力、载重能力和平顺性等较差。同时,电动代步车底盘本身是面向有人驾驶的,其存在着诸多问题,如转向盘等机械结构冗余、线控化程度低、续航能力差、电池系统不满足车规级需求、控制系统粗糙、执行元件精度低等,难以满足无人驾驶车辆实际的性能需求。更重要的是,当前用于改装的底盘五花八门、改装方案良莠不齐,这严重制约了未来特定场景无人驾驶车辆的网联化监测与运营。

\subsection{设计目标与意义}

希望以推动特定场景无人驾驶车辆大范围落地为近期目标,致力于无人驾驶车辆车规级“通用线控底盘”研发,为特定场景无人驾驶车辆用户提供通用的智能底盘平台;远景目标瞄向构建以通用底盘平台为基础的全工况、全维度、全周期的特定场景无人驾驶车辆监控与运营的大数据平台。

“通用线控底盘”:以“通用化”为核心特点,底盘被集成封装在一个扁平的密闭车身中,依据不同客户的造型需求,可搭载不同上装功能模块,如物流、快递、清扫、运输、甚至军警用特种装备等模块,成为适用于特定场景下各种功能的无人驾驶车辆。

\subsection{本报告章节安排}

本报告从AGV自动化小车缘由谈起,结合AGV在研究与应用现状与进展,收集整理了国内外在自动化导航小车的应用与不足,并比较分析后引出主要设计的目标与方向;接着就所提出的要求分别进行机械、电控从总体到模块分析的选型、设计,最后就整个机器人实现层面进行重难点分析与经济社会分析后,最终进行课程总结,主要内容包括以下几个部分:

第一章是主要分析与介绍AGV自动化小车的发展与应用前景,提出设计要求,并说明主要章节安排;

第二章是就所提出要求进行机械部分的结构设计,主要由总体设计、模块设计、;

第三章是结合当今国内发展的第四代自动化码头进行案例分析说明;

第四章是总结以上概述与所查询资料,总论未来自动化码头的发展趋势;

第五章是对参观、上课与论文书写的总结记录;

最后部分为本文参考文献。