\newpage
\section{绪论}
\subsection{集装箱码头的时代背景}

海运承载着世界贸易总量的2/3,在各种运输方式中,以低成本、运量大为主要特点并占据不可动摇的地位。作为沟通海运和其他运输方式的集装箱码头,就成为物流、资金流、信息流交汇的中心。

与此同时,船舶大型化,港口的装卸效率迫切需要不断提高.再加上人力成本,码头对环保、节能等方面的要求越来越高.自动化码头因其安全可靠性高、作业效率高、场地利用率高、环境友好、人力成本低等显著优点而成为普遍发展趋势,已在世界很多港口实现应用。由此可见,集装箱码头在经济全球化中扮演着重要的角色,因此自动化集装箱码头注定备受国际社会的关注。

\subsection{集装箱码头的发展原因}

to be continued

\subsection{本报告章节安排}

本文从当今世界港口货运物流谈起,结合集装箱码头的起源,收集整理了世界范围内三代典型集装箱自动化码头的方案,并相互比较得出对已有自动化码头进行相关问题以及解决方案的陈说;最后结合近五年来中国正在发展的全自动码头的几个案例概要分析阐述自动化码头与相关设备的发展方向,以期待有一个全面的论说,主要内容包括以下几个部分:

第一章是主要介绍集装箱自动化码头的时代背景与发展原因,并说明主要章节安排;

第二章是收集整理了世界范围内三代典型集装箱自动化码头的方案,并说明总结已有问题;

第三章是结合当今国内发展的第四代自动化码头进行案例分析说明;

第四章是总结以上概述与所查询资料,总论未来自动化码头的发展趋势;

第五章是对参观、上课与论文书写的总结记录;

最后部分为本文参考文献。